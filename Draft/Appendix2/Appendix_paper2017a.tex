\chapter{Variations in the multistream field}
\label{appendix:Eigen}

A second-order local variations of a scalar field $f$ is described by a Hessian. In a three-dimensional domain, the Hessian is given by \autoref{eq:Hess}. The geometry of the scalar field is classified by the Eigenvalues of the Hessian. The convex regions have at-most one maxima within the (3+1)-dimensional functional space. Projection of this closed region onto three-dimensional coordinate space also gives a closed surface in coordinate space. 


% An illustration of the projection is shown in \autoref{fig:check1d} for a simpler function $f(x)$ in one-dimensional domain. The eigenvalue criteria for regions are simplified: for instance, $\frac{\partial^2 f}{\partial x^2} < 0$ for convex region. Projection of these regions onto coordinate space is shown in the shaded regions. This is different from regions within a contour, which is the projection of the curve along which the function has a constant value. Boundaries of these two regions may, but not necessarily, intersect.    
% 
% \begin{figure}
% \begin{minipage}[t]{.99\linewidth}
%   \centering\includegraphics[width=8.cm]{fig25.pdf} 
% \end{minipage}\hfill
% \caption{Projections of regions of $f(x)$ from (1+1)-dimensional function space onto one-dimensional coordinate space. Convex regions and regions above a threshold of an arbitrary function $f(x)$ are shown. Both the regions intersect around a few maxima, but not universally.}
% \label{fig:check1d}
% \end{figure}



We treat $n_{str}$ approximately continuous, for which the Hessian is always symmetric. In this study we use the scalar field $n_{str}(\bf{x})$ inherently has discrete values like 1, 3, 5, and so on. The equation for numerical differentiation in the off-diagonal terms using Forward-difference method (using step-sizes of $\Delta x_i$ and $\Delta x_j$ along $i$ and $j$ respectively) is given in \autoref{eq:part1}. Notice that $\frac{\partial^2 f}{\partial x_i \partial x_j} = \frac{\partial^2 f}{\partial x_j \partial x_i}$, since RHS in \autoref{eq:part1} remains same. Hence the Hessian matrix in \autoref{eq:Hess} for the discrete scalar field $n_{str}$ is always numerically symmetric. Backward or central difference give similar results too. Smoothing of the multistream field further reduces any numerical noise in the Hessian eigenvalues.

\begin{equation}
\label{eq:part1}
\frac{\partial^2 f}{\partial x_i \partial x_j} = \frac{1}{\Delta x_i \Delta x_j}  \left[f_{i+1,j+1,k}-f_{i,j+1,k}-f_{i+1,j,k}+f_{i,j,k} \right]
\end{equation}

An integer-valued function, like the multistream field, is either constant or changes by a constant value in its real domain. In addition, the transitions in the multistream field are of multiples of 2, unless caustic surfaces are detected at the exact grid location. Consider $f_{i,j,k} = n$ at any grid point. Due to the property of multistream field, the values in the neighbourhood differ by a multiple of 2. That is,  $f_{i+1,j,k} = n+2p$, $f_{i,j+1,k} = n+2q$, $f_{i+1,j+1,k} = n+2r$, for some integers $p$, $q$ and $r$. Thus the second order variation of the multistream field reduces to \autoref{eq:part2}. 

\begin{equation}
\label{eq:part2}
\frac{\partial^2 f}{\partial x_i \partial x_j} = \frac{1}{\Delta x_i \Delta x_j}  \left[ 2r - 2p + 2q \right]
\end{equation}

Thus the numerical differentiation is independent of $n_{str}$ itself. It's important to note that this behaviour of the multistream field is independent of grid size. Also, the second order variation is a ratio of an even-number and the face area of the grid cube. The \autoref{eq:part2} becomes zero in a trivial case of $r = p = q = 0$, which corresponds to regions where $n_{str}$ is constant, including voids. In the non-trivial case, $r=(p+q)$, for non-zero $r$, $p$ and $q$. In the multistream grid, $2(p+q)$ could be considered as sum of variations in $n_{str}$ in the immediate neighbouring grid points. And $2r$ is the variation between next closest grid point, which is along the face-diagonal. 

On the other hand, mass density fields have sharp peaks at the multistream transitions. These peaks in the at the location of caustic are far less predictable, since the density fields become extremely noisy. For instance,\cite{Vogelsberger2011} show noisy peaks of varying magnitude at the at high resolutions of mean density near halo locations. At lower resolutions, these sharp peaks are smoothed out, hence giving the impression of a smooth field. \cite{Hahn2015a} show similar `ill-behaved' derivatives in velocity fields at the caustic locations, where the derivatives are infinite.   
