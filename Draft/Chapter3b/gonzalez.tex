\subsection{FINE \\ \hskip 0.75cm(Gonzalez \& Padilla)}
\label{section:gonzalez}

The filamentary structure in the cosmic web can be found by following the highest density paths between density peaks.
The Filament Identification using NodEs (\fine) method described in \citet{gonzalez2010} looks for filaments in halo or galaxy distributions.

The method requires halo/galaxy positions and masses (luminosities for galaxies), and we define as Nodes, the haloes/galaxies above a given mass/luminosity.
The mass of the nodes will define the scale of the filaments in the search. The smaller the node masses, the smaller the filaments that will be found between them.

The density field is computed using Voronoi Tessellations similar to \citet{schaapwey2000}.
The method looks first for a filament skeleton between any node pair by following the highest density path and a minimum separation; those two parameters characterize the filament quality.
Filament members are selected by binding energy in the plane perpendicular to the filament; this condition is associated to characteristic orbital times. However, if one assumes a fixed orbital timescale for all filaments, the resulting filament properties show only marginal changes, indicating that the use of dynamical information is not critical for this criterion.
Filaments detected using this method are in good agreement with \citet{colberg2005} who use  by-eye criteria.


In this comparison we define nodes as the haloes with masses above $5\times10^{13}M_{\odot}$, and the minimum density threshold for the skeleton search is $5$ times the mean Voronoi density.



%%%%%%%%%%References used
%@ARTICLE{2010MNRAS.407.1449G,
%   author = {{Gonz{\'a}lez}, R.~E. and {Padilla}, N.~D.},
%    title = "{Automated detection of filaments in the large-scale structure of the Universe}",
%  journal = {\mnras},
%archivePrefix = "arXiv",
%   eprint = {0912.0006},
% primaryClass = "astro-ph.CO",
% keywords = {large-scale structure of Universe},
%     year = 2010,
%    month = sep,
%   volume = 407,
%    pages = {1449-1463},
%      doi = {10.1111/j.1365-2966.2010.17015.x},
%   adsurl = {http://adsabs.harvard.edu/abs/2010MNRAS.407.1449G},
%  adsnote = {Provided by the SAO/NASA Astrophysics Data System}
%}
%
%@ARTICLE{2000A&A...363L..29S,
%   author = {{Schaap}, W.~E. and {van de Weygaert}, R.},
%    title = "{Continuous fields and discrete samples: reconstruction through Delaunay tessellations}",
%  journal = {\aap},
%   eprint = {astro-ph/0011007},
% keywords = {METHODS: N-BODY SIMULATIONS, METHODS: NUMERICAL, METHODS: STATISTICAL, COSMOLOGY: LARGE-SCALE STRUCTURE OF UNIVERSE},
%     year = 2000,
%    month = nov,
%   volume = 363,
%    pages = {L29-L32},
%   adsurl = {http://adsabs.harvard.edu/abs/2000A%26A...363L..29S},
%  adsnote = {Provided by the SAO/NASA Astrophysics Data System}
%}
%
%@ARTICLE{2005MNRAS.359..272C,
%   author = {{Colberg}, J.~M. and {Krughoff}, K.~S. and {Connolly}, A.~J.
%	},
%    title = "{Intercluster filaments in a {$\Lambda$}CDM Universe}",
%  journal = {\mnras},
%   eprint = {astro-ph/0406665},
% keywords = {methods: N-body simulations, cosmology: theory, dark matter, large-scale structure of Universe},
%     year = 2005,
%    month = may,
%   volume = 359,
%    pages = {272-282},
%      doi = {10.1111/j.1365-2966.2005.08897.x},
%   adsurl = {http://adsabs.harvard.edu/abs/2005MNRAS.359..272C},
%  adsnote = {Provided by the SAO/NASA Astrophysics Data System}
%}

