\label{section:simulation}

Each of the participants applied their web identification methods to the same Gadget-2~\citep{2005MNRAS.364.1105S} dark matter only $N$-body simulation, with a box size of 200 $\hmpc$ and $512^3$ particles. The $\Lambda$CDM cosmological parameters are taken from Planck \citep{2014A&A...571A..16P}: $h=0.68$, $\Omega_M = 0.31$, $\Omega_\Lambda = 0.69$, $n_s = 0.96$, and $\sigma_8 = 0.82$. Haloes in the simulation are identified using a standard FOF algorithm \citep{1985ApJ...292..371D}, with a linking length of $b=0.2$ and a minimum of 20 particles per halo. Fig. \ref{fig:den_slice} shows a thin slice through the density field and the halo population of this simulation.

The main output of the methods is the classification of the dark matter density field into one of four web components: knot, filament, wall and void. This classification is performed for either volume elements (e.g. the Hessian methods), dark matter mass elements (e.g. the phase-space methods), or for the haloes (e.g. the point process methods). The exact choice was left to the discretion of the authors to better reflect the procedure used in the studies employing those methods. 

Though the output format of the web identification methods may vary, each participant was asked to provide two datasets: the web identification tag defined on a regular grid with a $2\hmpc$ cell size ($100^3$ cells) and the web classification of each FOF halo. Most methods returned both datasets except for some of the point-process methods (MST, FINE), for which assigning a environment tag to each grid cell would not make sense. These return information regarding the filamentary environment of just the FOF haloes.

The simulation is made publicly available\footnote{http://data.aip.de/tracingthecosmicweb/ \\  doi:10.17876/data/2017\_1} for exploitation by interested parties. We have included the $z=0$ Gadget snapshots, the FOF halo catalogue as well as the output of each cosmic web method included in this work. Where available, each method's classification is returned on a regular grid. Included in the data set is also the FOF catalogue appended with the classification of each halo for each method. We encourage other methods not included in this paper, to use this data set as a bench mark of the community's current status.

%Comparison setup for each method:

