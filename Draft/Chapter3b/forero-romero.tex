\subsection{T-web: Tidal Shear Tensor\\ \hskip 0.75cm (Forero-Romero, Hoffman \& Gottl\"ober)}
\label{section:forero-romero}

This method \citep[T-web,][]{forero2009} works on density field grids
obtained either from numerical simulations or reconstructions from
redshift surveys. 

The method builds on the work by \cite{hahn2007a}. It
also uses the Hessian of the gravitational potential 
\begin{equation}
T_{\alpha\beta} = \frac{\partial^2\phi}{\partial x_\alpha\partial x_\beta},
\end{equation}
%
where the physical gravitational potential has been normalized by
$4\pi G\bar{\rho}$ so that $\phi$ satisfies the Poisson
equation
\begin{equation}
\nabla^2\phi=\delta,
\end{equation}
%
with $\delta$ the dimensionless matter overdensity, $G$ the
gravitational constant and $\bar{\rho}$ the average density of the
Universe.

This tidal tensor can be represented by a real symmetric $3\times 3$
matrix with eigenvalues $\lambda_1>\lambda_2>\lambda_3$ and
eigenvectors ${\bf e}_1$, ${\bf e}_2$ and ${\bf e}_3$. The eigenvalues
are indicators of orbital stability along the directions defined by
the eigenvectors. 

This method introduces a threshold $\lambda_{\rm th}$ to gauge the
strength of the eigenvalues of the tidal shear tensor. The number of
eigenvalues larger than the threshold is used to classify the cosmic
web into four kinds of environments: voids (3 eigenvalues smaller than
$\lambda_{\rm th}$), sheets (2), filaments (1) and knots (0).

In practice the density is interpolated over a grid using the particle
data and a Cloud-In-Cell scheme. The Poisson equation is solved in
Fourier space to obtain the potential over a grid. At each grid cell
the shear tensor is computed to obtain and store the corresponding
eigenvalues and eigenvectors. The grid cell has a size of $\sim
1h^{-1}$Mpc and the threshold is fixed to be $\lambda_{\th}=0.2$ as
suggested by previous studies that aim at capturing the visual
impression of the cosmic web \citep{forero2009}. 




%%%%%%%%%%References used
%@ARTICLE{2007MNRAS.375..489H,
%   author = {{Hahn}, O. and {Porciani}, C. and {Carollo}, C.~M. and {Dekel}, A.
%	},
%    title = "{Properties of dark matter haloes in clusters, filaments, sheets and voids}",
%  journal = {\mnras},
%   eprint = {astro-ph/0610280},
% keywords = {methods: N-body simulations , galaxies: haloes , cosmology: theory , dark matter , large-scale structure of Universe},
%     year = 2007,
%    month = feb,
%   volume = 375,
%    pages = {489-499},
%      doi = {10.1111/j.1365-2966.2006.11318.x},
%   adsurl = {http://adsabs.harvard.edu/abs/2007MNRAS.375..489H},
%  adsnote = {Provided by the SAO/NASA Astrophysics Data System}
%}



%@ARTICLE{2009MNRAS.396.1815F,
%   author = {{Forero-Romero}, J.~E. and {Hoffman}, Y. and {Gottl{\"o}ber}, S. and 
%	{Klypin}, A. and {Yepes}, G.},
%    title = "{A dynamical classification of the cosmic web}",
%  journal = {\mnras},
%archivePrefix = "arXiv",
%   eprint = {0809.4135},
% keywords = {methods: numerical , cosmology: large-scale structure of Universe},
%     year = 2009,
%    month = jul,
%   volume = 396,
%    pages = {1815-1824},
%      doi = {10.1111/j.1365-2966.2009.14885.x},
%   adsurl = {http://adsabs.harvard.edu/abs/2009MNRAS.396.1815F},
%  adsnote = {Provided by the SAO/NASA Astrophysics Data System}
%}





