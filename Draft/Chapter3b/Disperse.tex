\subsection{DisPerSE \\ \hskip 0.75cm(Sousbie)}
\label{section:Disperse}
\disperse{} is a formalism designed for analyzing the cosmic web, and in particular its filamentary network, 
on the basis of the topological structure of the cosmic mass distribution \citep{sousbie2011,sousbie2011b}. 
The elaborate framework of \disperse{} is based on three mathematical and computational pillars. These are 
Morse theory \citep{morse1934,milnor1963}, Discrete Morse theory \citep{forman1998,gyulassy2008} and the 
Delaunay Tessellation Field Estimator \dtfe{} \citep{schaapwey2000,weyschaap2009,cautun2011}. 
The formalism uses  three concepts in computational 
topology: Persistent Homology and Topological Simplification \citep{edelsbrunner2002,gyulassy2008,edelsbrunner2010}. 
These are used  for removal of noise and the selection of the significant morphological features from a discretely 
sampled cosmic mass distribution. 

\disperse{} analyzes and characterizes the cosmic web in terms of a spatial segmentation of space defined by the  
singularity structure of the cosmic mass distribution, the \emph{Morse complex}. 
The morphological components of the cosmic web are identified with the various $k$-dimensional manifolds that 
outline this uniquely defined segmentation. Filaments are identified with the ascending 
$1$-manifold. Voids, walls and clusters are identified with the ascending $3$-, $2$- and $0$-manifolds 
\citep[also see][]{aragon2010a,shivashankar2016}.  \disperse{} follows earlier 
applications of Morse theory to structural classification in astrophysical datasets \citep[][]{colombi2000}. 
The immediate precursor to \disperse{} is the skeleton formalism \citep{sousbie2008,sousbie2008b,sousbie2009}. 

%\medskip
Morse theory, which forms the basis for \disperse{}, looks at the singularity structure of the density field $f(x)$. It leads to 
the translation of the spatial distribution and connectivity of maxima, minima and saddle points in the density fields into a 
geometric segmentation of space that is known as the \emph{Morse complex}. This is a uniquely defined segmentation of space in a 
spatial tessellation of cells, faces, edges and nodes defined by the singularities and their connections by integral lines. 
The \emph{index} of a non-degenerate critical point is the number of negative eigenvalues of its Hessian: a \emph{minimum} of a 
field $f$ has index $0$, a \emph{maximum} has index $3$, while there are two types of \emph{saddles}, with index $1$  and $2$.  
Morse theory captures the connectivity of a field $f$ via the flowlines defined by the field gradient $\nabla f$, the \emph{integral lines}.
The field minima and maxima are the divergence and convergence points of these lines. It leads to a natural segmentation of space 
into distinct regions of space called ascending and descending manifolds. The \emph{ascending} $k$-manifold of a critical point $P$ defines 
the $k$-dimensional region of space defined by the set of points along integral lines that emanate from origin $P$. Conversely, the 
\emph{descending} $m$-manifold of a critical point $P$ is the $m$-dimensional region of space defined by the set of points along 
integral lines for which $P$ is the destination. 
%In all, there exist exactly $d$ different classes of ascending and descending 
%manifolds \citep[see e.g.][]{sousbie2011}. 

%\medskip
Since astronomical datasets (N-body simulations, galaxy catalogs, etc.) are discrete and intrinsically noisy tracers of the density field,  \disperse{} utilizes \emph{Discrete Morse theory}  \citep[see e.g.][]{forman1998}. It consists of a combinatorial formulation of Morse theory 
in terms of intrinsically discrete functions defined over a simplicial complex\footnote{In essence, a simplicial complex 
is a geometric assembly of cells, faces, edges and vertices marking a discrete map of the volume. Cells, faces, edges and 
nodes/vertices are $3-$, $2-$, $1$- and $0$-dimensional simplices.}. A well-known example 
is the Delaunay tessellation $\cal D$ \citep{weygaert1994,okabe2000}. \disperse{} uses the Delaunay tessellation, and 
specifically its role as functional basis in the \dtfe{} formalism \citep{schaapwey2000,weyschaap2009,cautun2011}. 
\disperse{} uses the \dtfe{} density estimate at each sample point, while 
it assigns a density value $f(\sigma_k)$ to each simplex $\sigma_k$ of the Delaunay simplicial complex $\cal D$. On the basis of 
these values, and the mutual connections between the various simplices, one may identify discrete simplicial analogues to the 
singularity points, gradient vector field, integral lines and Morse complex \citep[see e.g.][for a detailed 
treatment]{gyulassy2008,sousbie2011}. 

%\medskip
The finite sampling of the density field introduces noise into the detection of structural features. Instead of resorting to a simplistic feature-independent filtering 
operation, which tends to suppress or even annihilate real structural features, \disperse{} makes use of persistent homology \citep{edelsbrunner2002,edelsbrunner2010}. Topological persistence 
is the language that allows the identification of features according to their significance \citep{edelsbrunner2002}. 
Persistence theory defines a topological criterion for the birth and death of features, and the persistence of a feature, i.e. its significance, is
quantified according to the interval between its appearance and demise. For removal of insignificant 
features \disperse{} augments the persistence measurement with the \emph{topological simplification} of 
the discrete Morse complex \citep{edelsbrunner2002,gyulassy2008}, consisting of an ordered elimination of simplicial singularities and  their connections.

%\medskip
The final product of \disperse{} is a simplicial complex with appropriately adapted gradient lines and corresponding 
ascending manifolds. It provides a map of the morphological structures that make up the weblike 
arrangement of galaxies and mass elements on Megaparsec scales, identified in terms of the ascending 
manifolds of a discrete and topologically filtered Morse complex. Most outstanding is the filamentary network 
corresponding to the index $1$ ascending manifolds. 


%@ARTICLE{aragon2010a,
%   author = {{Arag{\'o}n-Calvo}, M.~A. and {Platen}, E. and {van de Weygaert}, R. and 
%	{Szalay}, A.~S.},
%    title = "{The Spine of the Cosmic Web}",
%  journal = {\apj},
%archivePrefix = "arXiv",
%   eprint = {0809.5104},
% keywords = {cosmology: theory, large-scale structure of universe, methods: numerical, surveys},
%     year = 2010,
%    month = nov,
%   volume = 723,
%    pages = {364-382},
%      doi = {10.1088/0004-637X/723/1/364},
%   adsurl = {http://adsabs.harvard.edu/abs/2010ApJ...723..364A},
%  adsnote = {Provided by the SAO/NASA Astrophysics Data System}
%}
%@MISC{cautun2011,
%   author = {{Cautun}, M.~C. and {van de Weygaert}, R.},
%    title = "{The DTFE public software: The Delaunay Tessellation Field Estimator code}",
% keywords = {Software},
%howpublished = {Astrophysics Source Code Library},
%     year = 2011,
%    month = may,
%archivePrefix = "arXiv",
%   eprint = {1105.0370},
% primaryClass = "astro-ph.IM",
%   adsurl = {http://adsabs.harvard.edu/abs/2011ascl.soft05003C},
%  adsnote = {Provided by the SAO/NASA Astrophysics Data System}
%}
%@ARTICLE{colombi2000,
%   author = {{Colombi}, S. and {Pogosyan}, D. and {Souradeep}, T.},
%    title = "{Tree Structure of a Percolating Universe}",
%  journal = {Physical Review Letters},
%   eprint = {astro-ph/0011293},
%     year = 2000,
%    month = dec,
%   volume = 85,
%    pages = {5515-5518},
%      doi = {10.1103/PhysRevLett.85.5515},
%   adsurl = {http://adsabs.harvard.edu/abs/2000PhRvL..85.5515C},
%  adsnote = {Provided by the SAO/NASA Astrophysics Data System}
%}
%@book{edelsbrunner2010,
%  title={Computational Topology: An Introduction},
%  author={Edelsbrunner, H. and Harer, J.},
%  isbn={9780821849255},
%  lccn={2009028121},
%  series={Applied mathematics},
%  url={http://books.google.at/books?id=MDXa6gFRZuIC},
%  year={2010},
%  publisher={American Mathematical Society}
%}
%@article{edelsbrunner2002,
%year={2002},
%issn={0179-5376},
%journal={Discrete Computat. Geom.},
%volume={28},
%number={4},
%doi={10.1007/s00454-002-2885-2},
%title={Topological Persistence and Simplification},
%url={http://dx.doi.org/10.1007/s00454-002-2885-2},
%publisher={Springer-Verlag},
%author={Edelsbrunner, H. and Letscher, J. and Zomorodian, A.},
%pages={511-533},
%language={English}
%}
%@article{forman1998,
%title = "Morse Theory for Cell Complexes",
%journal = "Advances in Mathematics",
%volume = "134",
%number = "1",
%pages = "90 - 145",
%year = "1998",
%note = "",
%issn = "0001-8708",
%doi = "http://dx.doi.org/10.1006/aima.1997.1650",
%url = "http://www.sciencedirect.com/science/article/pii/S0001870897916509",
%author = "Robin Forman"
%}
%@ARTICLE{gyulassy2008,
%   author = {{Gyulassy}, A.~G.},
%    title = "{Combinatorial construction of Morse-Smale complexes for data analysis and visualization}",
%  journal = {PhD Thesis},  
%     year = 2008
%}
%@article{milnor1963,
%    author = {{Milnor}, J.},
%    journal = {Journal of Mathematical Physics},
%    publisher = {Princeton Univ. Press, Princeton, New Jersey},
%    title = {Morse Theory},
%    year = {1963}
%}
%@Book{okabe2000,
%  author = 	 "Atsuyuki Okabe and Barry Boots and Kokichi Sugihara
%		  and Sung Nok Chiu",
%  title = 	 "Spatial tessellations: Concepts and applications of
%		  {V}oronoi diagrams",
%  publisher = 	 "Wiley",
%  year = 	 "2000",
%  series =	 "Probability and Statistics",
%  address =	 "NYC",
%  edition =	 "2nd",
%  note =	 "671 pages.",
%  OPTannote = 	 ""
%}
%@ARTICLE{pranav2017,
%   author = {{Pranav}, P. and {Edelsbrunner}, H. and {van de Weygaert}, R. and 
%	{Vegter}, G. and {Kerber}, M. and {Jones}, B.~J.~T. and {Wintraecken}, M.
%	},
%    title = "{The topology of the cosmic web in terms of persistent Betti numbers}",
%  journal = {\mnras},
%archivePrefix = "arXiv",
%   eprint = {1608.04519},
% keywords = {methods: data analysis, methods: numerical, methods: statistical, cosmology: theory, large-scale structure of Universe},
%     year = 2017,
%    month = mar,
%   volume = 465,
%    pages = {4281-4310},
%      doi = {10.1093/mnras/stw2862},
%   adsurl = {http://adsabs.harvard.edu/abs/2017MNRAS.465.4281P},
%  adsnote = {Provided by the SAO/NASA Astrophysics Data System}
%}
%@ARTICLE{schaapwey2000,
%  author = {Schaap, W.~E. and van de Weygaert, R.},
%  journal = {\aap},
%  year = {2000},
%  volume = {363},
%  pages = {L29},
%  timestamp = {2012.07.16}
%}
%@article{shivashankar2016,
%  title={Felix: A Topology Based Framework for Visual Exploration of Cosmic Filaments},
%  author={Shivashankar, Nithin and Pranav, Pratyush and Natarajan, Vijay and van de Weygaert, Rien and Bos, EG Patrick and Rieder, Steven},
%  journal={IEEE Transactions on Visualization and Computer Graphics},
%  volume={22},
%  number={6},
%  pages={1745--1759},
%  year={2016},
%  publisher={IEEE}
%}
%@ARTICLE{sousbie2008,
%   author = {{Sousbie}, T. and {Pichon}, C. and {Colombi}, S. and {Novikov}, D. and 
%	{Pogosyan}, D.},
%    title = "{The 3D skeleton: tracing the filamentary structure of the Universe}",
%  journal = {\mnras},
%archivePrefix = "arXiv",
%   eprint = {0707.3123},
% keywords = {cosmology: theory , dark matter , large-scale structure of Universe},
%     year = 2008,
%    month = feb,
%   volume = 383,
%    pages = {1655-1670},
%      doi = {10.1111/j.1365-2966.2007.12685.x},
%   adsurl = {http://adsabs.harvard.edu/abs/2008MNRAS.383.1655S},
%  adsnote = {Provided by the SAO/NASA Astrophysics Data System}
%}
%@ARTICLE{sousbie2011,
%   author = {{Sousbie}, T.},
%    title = "{The persistent cosmic web and its filamentary structure - I. Theory and implementation}",
%  journal = {\mnras},
%archivePrefix = "arXiv",
%   eprint = {1009.4015},
% keywords = {methods: data analysis, methods: numerical, galaxies: formation, galaxies: kinematics and dynamics, cosmology: observations, large-scale structure of% Universe},
%     year = 2011,
%    month = jun,
%   volume = 414,
%    pages = {350-383},
%      doi = {10.1111/j.1365-2966.2011.18394.x},
%   adsurl = {http://adsabs.harvard.edu/abs/2011MNRAS.414..350S},
%  adsnote = {Provided by the SAO/NASA Astrophysics Data System}
%}
%@ARTICLE{sousbie2011b,
%   author = {{Sousbie}, T. and {Pichon}, C. and {Kawahara}, H.},
%    title = "{The persistent cosmic web and its filamentary structure - II. Illustrations}",
%  journal = {\mnras},
%archivePrefix = "arXiv",
%   eprint = {1009.4014},
% keywords = {methods: data analysis, galaxies: formation, galaxies: kinematics and dynamics, cosmology: observations, dark matter, large-scale structure of Univer%se},
%     year = 2011,
%    month = jun,
%   volume = 414,
%    pages = {384-403},
%      doi = {10.1111/j.1365-2966.2011.18395.x},
%   adsurl = {http://adsabs.harvard.edu/abs/2011MNRAS.414..384S},
%  adsnote = {Provided by the SAO/NASA Astrophysics Data System}
%}
%@PHDTHESIS{weygaert1991,
%   author = {{van de Weygaert}, M.~A.~M.},
%    title = "{Voids and the geometry of large scale structure}",
%   school = {Ph.~D.~thesis, University of Leiden (1991)},
%     year = 1991,
%    month = September,
%   adsurl = {http://adsabs.harvard.edu/abs/1991PhDT........84V},
%  adsnote = {Provided by the SAO/NASA Astrophysics Data System}
%}
%@ARTICLE{weygaert1994,
%   author = {{van de Weygaert}, R.},
%    title = "{Fragmenting the Universe. 3: The constructions and statistics of 3-D Voronoi tessellations}",
%  journal = {\aap},
% keywords = {Astronomical Models, Computational Astrophysics, Cosmology, Galactic Clusters, Galactic Evolution, Mathematical Models, Statistical Analysis, Three D%imensional Models, Universe, Algorithms, Boundary Conditions, Computerized Simulation, Galactic Nuclei, Histograms, Monte Carlo Method, Poisson Equation, Spatial %Distribution},
%     year = 1994,
%    month = mar,
%   volume = 283,
%    pages = {361-406},
%   adsurl = {http://adsabs.harvard.edu/abs/1994A\%26A...283..361V},
%  adsnote = {Provided by the SAO/NASA Astrophysics Data System}
%}
%@ARTICLE{weyschaap2009,
%  author = {{van de Weygaert}, R. and {Schaap}, W.},
%  title = {{The Cosmic Web: Geometric Analysis}},
%  year = {2009},
%  volume = {665},
%  pages = {291-413},
%  adsnote = {Provided by the SAO/NASA Astrophysics Data System},
%  adsurl = {http://adsabs.harvard.edu/abs/2009LNP...665..291V},
%  booktitle = {Data Analysis in Cosmology},
%  doi = {10.1007/978-3-540-44767-211},
%  editor = {{V.~J.~Mart{\'{\i}}nez, E.~Saar, E.~Mart{\'{\i}}nez-Gonz{\'a}lez,
%	\& M.-J.~Pons-Border{\'{\i}}a}},
%  file = {:net/plato/data/users/cautun/Papers/DTFE/The cosmic web Geometric Analysis - Weygaert & Schaap 2007.pdf:PDF},
%  owner = {cautun},
%  series = {Lecture Notes in Physics, Berlin Springer Verlag},
%  timestamp = {2011.04.06}
%}



