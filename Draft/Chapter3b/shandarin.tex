\subsection{Multi-Stream Web Analysis (MSWA)\\ \hskip 0.85cm (Shandarin \& Ramachandra)}
\label{section:shandarin}

The growth of CDM density perturbations results in the emergence of regions where the velocity has distinct multiple values.
These regions are also of high densities. The DM web can be viewed as a multi stream field. For example, voids are the regions 
where the velocity field has a single value. This is because no gravitationally bound DM object can form prior to origin of shell crossing, which 
corresponds to the formation of regions with at least three streams. Three-stream flows associated with Zel'dovich's pancakes are
gravitationally bound only in one direction, roughly perpendicular to the pancake. Filaments are bound in two directions orthogonal to the
filament, and haloes are of course fully bound.
The generic geometrical structures of the web, i.e. walls aka pancakes, filaments and haloes, 
cannot be uniquely defined by any particular threshold of the multi stream field, which is also true for the density field.
Generally all three-stream regions belong to the walls, but the transition to the filaments and to the haloes may occur at different levels.

The multi stream field can be easily computed from the final and initial coordinates of the particles in cosmological $N$-body simulations
\citep{ abel2012,shandarin2012}. 
A cold collisionless matter represents an extremely thin three-dimensional sheet called a Lagrangian submanifold in six-dimensional space 
made by  three  initial and three final coordinates at a chosen state of the simulation. Similarly to the three-dimensional sheet in six-dimensional phase space, the
Lagrangian submanifold contains full dynamic information about the system. It needs to be tetrahedralized only once by using initial positions of the particles
on a regular grid as vertices of the tetrahedra. During the following evolution the tessellation remains intact. It always remains continuous, and
its projection in 3D coordinate space fully tiles it at least by one layer in voids and many times in the web regions.
The tetrahedra during the evolution are deformed, but the deformation has no effect on the connectivity assignments  between the particles.
The number of streams can be computed on an arbitrary set  of spatial points by simply counting how many tetrahedra contain a given point. 
The first study of the multi-stream environment of DM haloes has been recently described in \citet{Ramachandara_Shandarin:15}.


Delineating the web components (walls, filaments and haloes) in multi-stream fields is not straightforward, and could be done using various approaches. By studying the scaling of multi-stream variation around dark-matter haloes, \cite{Ramachandara_Shandarin:15} showed that the geometries of structures change from sheets to filaments at a multi-stream value of $n_{\rm str} \gtrsim 17$. The next transition from filaments to knots is seen at around $n_{\rm str} \gtrsim 90$ -- which also roughly corresponds to the virial mass density $\Delta_{\rm vir} = 200$. These thresholds are heuristic -- the analysis may have to be repeated for different simulations for the calibration of thresholds. On the other hand, local Hessian-based geometric methods were recently used to identify multi-stream structures by \cite{Ramachandra2017}. This approach hints towards a portrait of structures in multi-stream fields that are free of ad hoc thresholds. For instance, the haloes could be identified simply as convex surfaces enclosing a local maxima of the multi-stream field (Ramachandra \& Shandarin, in preparation).

A summary of the classification scheme used for this comparison project is as follows: Voids are simply the regions with $n_{str} = 1$. The web components are delineated by utilizing the first approach of calibrating thresholds, i.e, sheets: $3 \leq n_{\rm str} < 17$, filaments: $ 17 \leq n_{\rm str} < 90$ and knots: $n_{\rm str} \geq 90$.


% %%%%%%%%%%References used
% @ARTICLE{Shandarin_etal:12,
%    author = {{Shandarin}, S. and {Habib}, S. and {Heitmann}, K.},
%     title = "{Cosmic web, multistream flows, and tessellations}",
%   journal = {\prd},
% archivePrefix = "arXiv",
%    eprint = {1111.2366},
%  primaryClass = "astro-ph.CO",
%  keywords = {Galaxy groups clusters and superclusters, large scale structure of the Universe, Superclusters, large-scale structure of the Universe, Cosmology, Origin and formation of the Universe},
%      year = 2012,
%     month = apr,
%    volume = 85,
%    number = 8,
%       eid = {083005},
%     pages = {083005},
%       doi = {10.1103/PhysRevD.85.083005},
%    adsurl = {http://adsabs.harvard.edu/abs/2012PhRvD..85h3005S},
%   adsnote = {Provided by the SAO/NASA Astrophysics Data System}
% }
% @article{Abel_etal:12,
% author = {{Abel}, T. and {Hahn}, O. and {Kaehler}, R.},
% title = "{Tracing the dark matter sheet in phase space}",
% journal = {MNRAS},
% archivePrefix = "arXiv",
% eprint = {1111.3944},
% primaryClass = "astro-ph.CO",
%  keywords = {methods: numerical, galaxies: formation, cosmology: theory, dark matter, large-scale structure of Universe},
% year = 2012,
% month = nov,
% volume = 427,
% pages = {61-76},
% doi = {10.1111/j.1365-2966.2012.21754.x},
% adsurl = {http://adsabs.harvard.edu/abs/2012MNRAS.427...61A},
% adsnote = {Provided by the SAO/NASA Astrophysics Data System}
% }
% @ARTICLE{Ramachandara_Shandarin:15,
% author = {Ramachandra, Nesar S. and Shandarin, Sergei F.}, 
% title = {Multi-stream portrait of the cosmic web},
% journal = {Monthly Notices of the Royal Astronomical Society},
% volume = {452}, 
% number = {2}, 
% pages = {1643-1653}, 
% year = {2015}, 
% doi = {10.1093/mnras/stv1389}, 
% eprint = {http://mnras.oxfordjournals.org/content/452/2/1643.full.pdf+html}, 
% } 




