
\subsection{MMF/Nexus: the Multiscale Morphology Filter \\ \hskip 0.75cm(Arag\'on-Calvo, Cautun, van de Weygaert \& Jones)} 
The MMF/Nexus Multiscale Morphology Filter technique \citep[][]{aragon2007,aragon2010b,cautun2013,cautun2014,aragon2014} performs the morphological 
identification of the cosmic web using a \textit{Scale-Space formalism} that ensures the detection of structures present at all scales. The formalism 
consists of a fully adaptive framework for classifying the matter distribution on the basis of local variations in the density field, velocity 
field or gravity field encoded in the Hessian matrix in these scales. Subsequently, a set of morphological filters is used to classify the spatial 
matter distribution into three basic components, the clusters, filaments and walls that constitute the cosmic web. The final product of the 
procedure is a complete and unbiased characterization of the cosmic web components, from the prominent features present in overdense regions 
to the tenuous networks pervading the cosmic voids.

Instrumental for this class of MMF cosmic web identification methods is that it simultaneously pays heed to two principal 
aspects characterizing the weblike cosmic mass distribution. The first aspect invokes the Hessian of the corresponding 
fields to probe the existence and identity of the mostly anisotropic structural components of the cosmic web. 
The second, equally important, aspect uses a scale-space analysis to probe the multiscale character of the cosmic mass distribution, 
the product of the hierarchical evolution and buildup of structure in the Universe. 

The Scale Space representation of a data set consists of a sequence of copies of the data having different resolutions 
\citep{florack1992,lindeberg1998}. A feature searching algorithm is applied to all of these copies, and the features are extracted in a 
scale independent manner by suitably combining the information from all copies. A prominent application of Scale Space analysis involves 
the detection of the web of blood vessels in a medical image \citep{sato1998,li2003}. The similarity to the structural patterns seen on 
Megaparsec scales is suggestive. The Multiscale Morphology Filter has translated, extended and optimized the Scale Space technology to 
identifying the principal characteristic structural elements in the cosmic mass and galaxy distribution.  
%The basic setup of MMF/Nexus is that of defining a four-dimensional scale-space representation of the input field $f(\Vector{x})$. In nearly 
%all implementations this is achieved by means of a Gaussian filtering of $f(\Vector{x})$ over a set of scales $[R_0,R_1,...,R_N]$. 
%%\begin{equation}
%%    f_{R_n}(\Vector{x}) = \int \frac{{\rm d}^3k}{(2\pi)^3} e^{-k^2R_n^2/2} \hat{f}(\Vector{k})  e^{i\Vector{k}\cdot\Vector{x}} ,
%%    \label{eq:filtered_field}
%%\end{equation}
%%\noindent where $\hat{f}(\Vector{k})$ is the Fourier transform of the input field $f(\Vector{x})$. 
%Subsequently, the Hessian $H_{ij,R_n}(\Vector{x})$ of the filtered field is computed, including a 
%normalization term for the scale, $R_n^2$. 
% \begin{eqnarray}
%    H_{ij,R_n}(\Vector{x})&\,=\,&R_n^2 \; \frac{\partial^2f_{R_n}(\Vector{x})}{\partial x_i\partial x_j}\,.
%%    \ \\
%%    \hat{H}_{ij,R_n}(\Vector{k})&\,=\,& -k_ik_j R_n^2 \hat{f}(\Vector{k}) e^{-k^2R_n^2/2}\,.\nonumber
%    \label{eq:hessian_general}
%\end{eqnarray}
%%\noindent In this, the $R_n^2$ term serves as a renormalization factor that has to do with the multiscale nature of the algorithm. 
%%
%The morphological information of the analyzed mass distribution is contained in the local geometry as specified 
%by the eigenvalues of the Hessian matrix, $\lambda_1 \le \lambda_2 \le \lambda_3$. On the basis of the corresponding 
%expected geometric behaviour, to every point $\Vector{x}$ the eigenvalues are used to assign a knot/cluster, filament and 
%wall characteristic. This is accomplished by means of a set of morphology filters on the resulting eigenvalue fields 
%\citep[see][]{aragon2007,cautun2013}. 
%
%The product of the morphology filter operation consists of the assignment of the environment signature $\mathcal{S}_{R_n}(\Vector{x})$, 
%to each volume element, at each scale represented in scale space. Subsequently, for each point, the environmental signatures 
%computed for each scale are combined to obtain a scale independent signature $\mathcal{S}(\Vector{x})$. It is defined as the 
%the maximum signature over all scales.
%%\begin{equation}
%%    \mathcal{S}(\Vector{x}) = \max_{\rmn{levels\;}n} \mathcal{S}_{R_n}(\Vector{x})\,.
%%    \label{eq:total_response}
%%\end{equation} 
%
%The final step in the MMF/Nexus procedure involves the use of criteria to find the threshold signature that discriminates 
%between valid and invalid detections. Signature values larger than the threshold correspond to real structures while the rest 
%are spurious detections. The different implementations and versions of the MMF/Nexus technique often differ in the 
%definition of the threshold values. 
%
The final outcome of the MMF/Nexus procedure is a field which at each location $\Vector{x}$ 
specifies what the local morphological signature is, cluster node, filaments, wall or void. 
The MMF/Nexus algorithms perform the environment detection by applying the above steps first to knots, then to filaments and 
finally to walls. Each volume element is assigned a single environment characteristic by requiring that filament regions 
cannot be knots and that wall regions cannot be either knots or filaments. The remaining regions are classified as voids.

Following the basic version of the MMF technique introduced by \cite{aragon2007}, it was applied to the analysis of the cosmic web 
in simulations of cosmic structure formation \citep{aragon2010b} and for finding filaments and galaxy-filament alignments in the SDSS galaxy 
distribution \citep{jones2010}. The principal technique, and corresponding philosophy, has subsequently branched into several 
further elaborations and developments. In this survey, we describe the Nexus formalism developed by \cite{cautun2013} and the 
MMF2 method developed by \cite{aragon2014}. Nexus has extended the MMF formalism to a substantially wider range of 
physical agents involved in the formation of the cosmic web, along with a substantially firmer foundation for the 
criteria used in identifying the various weblike structures. MMF-2 not only focusses on the multiscale nature 
of the cosmic web itself, but also addresses the nesting relations of the hierarchy. 

\subsubsection{\nexus{} \\ \hskip 0.9cm (Cautun, van de Weygaert \& Jones)}
\label{section:nexus}
The \nexus{} version of the MMF/Nexus formalism \citep{cautun2013,cautun2014} builds upon 
the original Multiscale Morphology Filter \citep{aragon2007,aragon2010b} algorithm and was developed with the goal of obtaining 
a more physically motivated and robust method. 

\nexus{} is the principal representative of the full \Nexus{} suite of cosmic web identifiers \citep[see][]{cautun2013}. 
These include the options for corresponding multiscale morphology identifiers on the basis of the raw density, the 
logarithmic density, the velocity divergence, the velocity shear and tidal force field. \Nexus{} has incorporated these 
options in a versatile code for the analysis of cosmic web structure and dynamics following the realization that they 
are significant physical influences in shaping the cosmic mass distribution into the complexity of the cosmic web. 

\nexus{} takes as input a regularly sampled density field. In a first step, the input field is Gaussian smoothed 
over using a \logFilter{} filter \citep[see][]{cautun2013} that is applied over a set of scales $[R_0,R_1,...,R_N]$, 
with $R_n=2^{n/2}R_0$. 
%For each of these scales, \nexus{} computes the eigenvalues of the Hessian matrix of the 
%smoothed density field. 
%Using the Hessian eigenvalues, following the procedure outlined above, 
\nexus{} then computes an environmental signature for 
each volume element. 

The \Nexus{} suite of MMF identifiers pays particular attention to the key aspect of setting the detection 
thresholds for the environmental signature. Physical criteria are used to determine a detection threshold. All points 
with signature values above the threshold are valid structures. For knots, the threshold is given by the requirement 
that most knot-regions should be virialized. For filaments and walls, the threshold is determined on the basis of the 
change in filament and wall mass as a function of signature. The peak of the mass variation with signature delineates 
the most prominent filamentary and wall features of the cosmic web.

For the \nexus{} implementation, the Delaunay Tessellation Field Estimator \dtfe{} method \citep{schaapwey2000,weyschaap2009} is  
used to interpolate the dark matter particle distribution to a continuous density field defined on a regular grid of size 
$600^3$ (grid spacing of $0.33\hmpc$). \nexus{} was applied to the resulting density field using a set of 7 smoothing scales 
from $0.5$ to $4\hmpc$ (in increments of $\sqrt{2}$ factors). This resulted in an environment tag for each grid cell that, in a second step, 
was down sampled to the target $100^3$ grid using a mass weighted selection scheme. For each cell of the coarser grid, we computed the mass 
fraction in each environment using all the fine level cells ($6^3$ in total) that overlap the coarser one. Then, the coarser cell was assigned 
the environment corresponding to the largest mass fraction. Each FOF halo was assigned the web tag corresponding to the fine grid cell in 
which the halo centre was located.

\subsubsection{MMF-2: Multiscale Morphology Filter-2\\ \hskip 0.9cm (Arag\'on-Calvo)}
\label{section:MMF}
%The Multiscale Morphology Filter \citep[MMF,][]{aragon2007,aragon2010b,cautun2013,aragon2014} method is a fully adaptive algorithm that classifies 
%the matter distribution into three basic morphologies: cluster, filament and wall. The morphological classification is based on 
%the local variations of the density field encoded in the Hessian matrix. 
The MMF-2 implementation of the MMF formalism differs from the \Nexus{} formalism in that it focusses on the multiscale character 
of the initial density field, instead of that of the evolved mass distribution. 
In order to account for hierarchical nature of the cosmic web, MMF-2 introduces the concept of \textit{hierarchical space} 
\citep{aragon2010, aragon2013}. While the conventional scale-space approach emphasizes the scale of the structures, it does 
not addresss their nesting relations. To accomplish this, MMF-2 resorts to the alternative of \textit{hierarchical space} 
\citep{aragon2010, aragon2013, aragon2014}. 

\textit{Hierarchical space} is created in the 
first step in the MMF-2 procedure \citep{aragon2010, aragon2013}. It is obtained by Gaussian-smoothing the initial conditions, 
and in principal concerns a continuum covering the full range of scales in the density field. For practical purposes however, a 
small set of linear-regime smoothed initial conditions is generated. Subsequently, by means of an N-body code these conditions 
are gravitationally evolved to the present time. 

By applying to linear-regime smoothing, \textit{hierarchical space} involves density field Fourier modes that are independent. This 
allows the user to target specific scales in the density field before Fourier mode-mixing occurs. The subsequent gravitational 
evolution of these smooth initial conditions results in a mass distribution that contains all the anisotropic features of the Cosmic Web, 
while it lacks the small-scale structures below the smoothing scale. Dense haloes corresponding to these small scales are absent. This reduces the 
dynamic range in the density field and greatly limits the contamination produced by dense haloes in the identification of filaments 
and walls. 

In line with the MMF procedure, for each realisation in the hierarchical space a set of 
morphology filters is applied, defined by ratios between the eigenvalues of the Hessian matrix 
\citep[$\lambda_1 < \lambda_2 < \lambda_3$, see][]{aragon2007}.  It also involves the applications of a threshold to the response from 
each morphology filter. This leads to a  final product consisting of a set of binary masks sampled on a regular grid indicating 
which voxels belong to a given morphology at a given hierarchical level. 

%%%%%%%%%%References used
%
%@ARTICLE{aragon2007,
%   author = {{Arag{\'o}n-Calvo}, M.~A. and {Jones}, B.~J.~T. and {van de Weygaert}, R. and 
%	{van der Hulst}, J.~M.},
%    title = "{The multiscale morphology filter: identifying and extracting spatial patterns in the galaxy distribution}",
%  journal = {\aap},
%archivePrefix = "arXiv",
%   eprint = {0705.2072},
% keywords = {cosmology: theory, large-scale structure of Universe, methods: statistical, surveys},
%     year = 2007,
%    month = oct,
%   volume = 474,
%    pages = {315-338},
%      doi = {10.1051/0004-6361:20077880},
%   adsurl = {http://adsabs.harvard.edu/abs/2007A%26A...474..315A},
%  adsnote = {Provided by the SAO/NASA Astrophysics Data System}
%}
%
%@ARTICLE{aragon2010,
%   author = {{Aragon-Calvo}, M.~A. and {van de Weygaert}, R. and {Araya-Melo}, P.~A. and 
%	{Platen}, E. and {Szalay}, A.~S.},
%    title = "{Unfolding the hierarchy of voids}",
%  journal = {\mnras},
%archivePrefix = "arXiv",
%   eprint = {1002.1503},
% keywords = {methods: N-body simulations, methods: data analysis, techniques: image processing, large-scale structure of Universe},
%     year = 2010,
%    month = may,
%   volume = 404,
%    pages = {L89-L93},
%      doi = {10.1111/j.1745-3933.2010.00841.x},
%   adsurl = {http://adsabs.harvard.edu/abs/2010MNRAS.404L..89A},
%  adsnote = {Provided by the SAO/NASA Astrophysics Data System}
%}
%
%@ARTICLE{aragon2010b,
%   author = {{Arag{\'o}n-Calvo}, M.~A. and {van de Weygaert}, R. and {Jones}, B.~J.~T.
%	},
%    title = "{Multiscale phenomenology of the cosmic web}",
%  journal = {\mnras},
%archivePrefix = "arXiv",
%   eprint = {1007.0742},
% keywords = {methods: data analysis, methods: numerical, cosmology: theory, large-scale structure of Universe},
%     year = 2010,
%    month = nov,
%   volume = 408,
%    pages = {2163-2187},
%      doi = {10.1111/j.1365-2966.2010.17263.x},
%   adsurl = {http://adsabs.harvard.edu/abs/2010MNRAS.408.2163A},
%  adsnote = {Provided by the SAO/NASA Astrophysics Data System}
%}
%
%@ARTICLE{aragon2013,
%   author = {{Aragon-Calvo}, M.~A. and {Szalay}, A.~S.},
%    title = "{The hierarchical structure and dynamics of voids}",
%  journal = {\mnras},
%archivePrefix = "arXiv",
%   eprint = {1203.0248},
% keywords = {methods: data analysis, large-scale structure of Universe},
%     year = 2013,
%    month = feb,
%   volume = 428,
%    pages = {3409-3424},
%      doi = {10.1093/mnras/sts281},
%   adsurl = {http://adsabs.harvard.edu/abs/2013MNRAS.428.3409A},
%  adsnote = {Provided by the SAO/NASA Astrophysics Data System}
%}
%
%
%@ARTICLE{aragon2014,
%   author = {{Aragon-Calvo}, M.~A. and {Yang}, L.~F.},
%    title = "{The hierarchical nature of the spin alignment of dark matter haloes in filaments}",
%  journal = {\mnras},
%archivePrefix = "arXiv",
%   eprint = {1303.1590},
% keywords = {methods: data analysis, galaxies: formation, large-scale structure of Universe},
%     year = 2014,
%    month = may,
%   volume = 440,
%    pages = {L46-L50},
%      doi = {10.1093/mnrasl/slu009},
%   adsurl = {http://adsabs.harvard.edu/abs/2014MNRAS.440L..46A},
%  adsnote = {Provided by the SAO/NASA Astrophysics Data System}
%}
%
%@ARTICLE{cautun2013,
%   author = {{Cautun}, M. and {van de Weygaert}, R. and {Jones}, B.~J.~T.
%	},
%    title = "{NEXUS: tracing the cosmic web connection}",
%  journal = {\mnras},
%archivePrefix = "arXiv",
%   eprint = {1209.2043},
% keywords = {methods: data analysis, techniques: image processing, cosmology: theory, large-scale structure of Universe},
%     year = 2013,
%    month = feb,
%   volume = 429,
%    pages = {1286-1308},
%      doi = {10.1093/mnras/sts416},
%   adsurl = {http://adsabs.harvard.edu/abs/2013MNRAS.429.1286C},
%  adsnote = {Provided by the SAO/NASA Astrophysics Data System}
%}2009LNP...665..291V
%
%@ARTICLE{cautun2014,
%   author = {{Cautun}, M. and {van de Weygaert}, R. and {Jones}, B.~J.~T. and 
%	{Frenk}, C.~S.},
%    title = "{Evolution of the cosmic web}",
%  journal = {\mnras},
%archivePrefix = "arXiv",
%   eprint = {1401.7866},
% keywords = {methods: data analysis, cosmology: theory, large-scale structure of Universe},
%     year = 2014,
%    month = jul,
%   volume = 441,
%    pages = {2923-2973},
%      doi = {10.1093/mnras/stu768},
%   adsurl = {http://adsabs.harvard.edu/abs/2014MNRAS.441.2923C},
%  adsnote = {Provided by the SAO/NASA Astrophysics Data System}
%}
%@ARTICLE{florack1992,
%    author = {Luc M J Florack and Bat M Ter Haar Romeny and Jan J Koenderink and Max A Viergever},
%    title = {Scale and the differential structure of images},
%    journal = {Image and Vision Computing},
%    year = {1992},
%    volume = {10},
%    pages = {376--388}
%}
%
%@ARTICLE{jones2010,
%   author = {{Jones}, B.~J.~T. and {van de Weygaert}, R. and {Arag{\'o}n-Calvo}, M.~A.
%	},
%    title = "{Fossil evidence for spin alignment of Sloan Digital Sky Survey galaxies in filaments}",
%  journal = {\mnras},
%archivePrefix = "arXiv",
%   eprint = {1001.4479},
% keywords = {methods: data analysis, cosmology: observations, cosmology: theory, large-scale structure of Universe},
%     year = 2010,
%    month = oct,
%   volume = 408,
%    pages = {897-918},
%      doi = {10.1111/j.1365-2966.2010.17202.x},
%   adsurl = {http://adsabs.harvard.edu/abs/2010MNRAS.408..897J},
%  adsnote = {Provided by the SAO/NASA Astrophysics Data System}
%}
%@article{lindeberg1998,
% author = {Lindeberg, T.},
%title = {Feature Detection with Automatic Scale Selection},
% journal = {Int. J. Comput. Vision},
% issue_date = {Nov. 1998},
% volume = {30},
% number = {2},
% month = nov,
% year = {1998},
% issn = {0920-5691},
% pages = {79--116},
% numpages = {38},
% acmid = {305298},
% publisher = {Kluwer Academic Publishers}
%} 
%
%@ARTICLE{schaapwey2000,
%   author = {{Schaap}, W.~E. and {van de Weygaert}, R.},
%    title = "{Continuous fields and discrete samples: reconstruction through Delaunay tessellations}",
%  journal = {\aap},
%   eprint = {astro-ph/0011007},
% keywords = {METHODS: N-BODY SIMULATIONS, METHODS: NUMERICAL, METHODS: STATISTICAL, COSMOLOGY: LARGE-SCALE STRUCTURE OF UNIVERSE},
%     year = 2000,
%    month = nov,
%   volume = 363,
%    pages = {L29-L32},
%   adsurl = {http://adsabs.harvard.edu/abs/2000A%26A...363L..29S},
%  adsnote = {Provided by the SAO/NASA Astrophysics Data System}
%}
%
%@INPROCEEDINGS{weyschaap2009,
%   author = {{van de Weygaert}, R. and {Schaap}, W.},
%    title = "{The Cosmic Web: Geometric Analysis}",
%booktitle = {Data Analysis in Cosmology},
%     year = 2009,
%   series = {Lecture Notes in Physics, Berlin Springer Verlag},
%   volume = 665,
%   editor = {{Mart{\'{\i}}nez}, V.~J. and {Saar}, E. and {Mart{\'{\i}}nez-Gonz{\'a}lez}, E. and 
%	{Pons-Border{\'{\i}}a}, M.-J.},
%    pages = {291-413},
%      doi = {10.1007/978-3-540-44767-2_11},
%   adsurl = {http://adsabs.harvard.edu/abs/2009LNP...665..291V},
%  adsnote = {Provided by the SAO/NASA Astrophysics Data System}
%}
%
%
%
%
%
%
