
\chapter{General Introduction}


\section{Large Scale structure}

\section{Observation history}

\section{Significance and Impact of the Cosmic Web}
Understanding the nature of the cosmic web is important for a variety of reasons. Quantitative measures of the cosmic web 
may provide information about the dynamics of gravitational structure formation, the background cosmological model, the 
nature of dark matter and ultimately the formation and evolution of galaxies. Since the cosmic web defines the fundamental spatial organization of matter and galaxies on scales of one to tens of Megaparsecs, its structure probes a wide variety of scales, form the linear to the nonlinear regime. This suggests that quantification of the cosmic web at these scales should provide a significant amount of 
information regarding the structure formation process. As yet, we are only at the beginning of systematically exploring the various 
structural aspects of the cosmic web and its components towards gaining deeper insights into the emergence of spatial 
complexity in the Universe \citep[see e.g.][]{Cautun2014a}. 

The cosmic web is also a rich source of information regarding the underlying cosmological model. The evolution, structure and dynamics of the 
cosmic web are to a large extent dependent on the nature of dark matter and dark energy. As the evolution of the cosmic web 
is directly dependent on the rules of gravity, each of the relevant cosmological variables will leave its imprint on the 
structure, geometry and topology of the cosmic web and the relative importance of the structural elements of the 
web, i.e. of filaments, walls, cluster nodes and voids. A telling illustration of this is the fact that  void regions of the cosmic web offer one of the cleanest probes and measures of dark energy as well as tests of gravity and General Relativity. Their structure and shape, as well as mutual alignment, are direct 
reflections of dark energy \citep{Parklee2007,Platen2008,Leepark2009,Lavaux2010,Lavaux2012,Bos2012,Sutter2014a,Pisani2015}. Given that the measurement of cosmological parameters depends on the observer's web environment \citep[e.g.][]{Wojtak2014}, one of our 
main objectives is to develop means of exploiting our measures of filament structure and dynamics, and the connectivity 
characteristics of the weblike network, towards extracting such cosmological information. 

Perhaps the most prominent interest in developing more objective and quantitative measures of large-scale cosmic web 
environments concerns the environmental influence on the formation and evolution of galaxies, and the dark matter 
halos in which they form \citep[see e.g.][]{Hahn2007b,Hahnphd2009,Cautun2014}. The canonical example of such 
an influence is that of the origin of the rotation of galaxies: the same tidal forces responsible for the torquing of collapsing 
protogalactic halos \citep{Hoyle1951,Peebles1969,Doroshkevich1970} are also directing the anisotropic contraction of matter in 
the surroundings. We may therefore expect to find an alignment between galaxy orientations and large scale filamentary 
structure, which indeed currently is an active subject of investigation \citep[e.g.][]{Aragon2007a,Leepen2000,Jones2010,Codis2012,Tempel2012,2013MNRAS.428.2489L,
tempel2013,Trowland2013,Trowlandphd2013,Aragon2014,2016MNRAS.457..695P,Hirv2017,2017MNRAS.464.4666G}. Some studies even claim this implies an instrumental 
role of filamentary and other weblike environments in determining 
the morphology of galaxies \citep[see e.g.][for a short review]{Pichon2016}. Indeed, the direct impact of the structure and 
connectivity of filamentary web on the star formation activity of forming galaxies has been convincingly demonstrated 
by \citet[][see also \citealt{2009ApJ...703..785D,2015MNRAS.449.2087D,2015MNRAS.454..637G,aragon2016}]{Dekel2008}. Such studies point out the instrumental importance of the filaments as transport conduits of cold 
gas on to the forming galaxies, and hence the implications of the topology of the network in determining the evolution and 
final nature. Such claims are supported by a range of observational findings, of which the morphology-density relation 
\citep{Dressler1980} is best known as relating intrinsic galaxy properties with the cosmic environment in which the 
galaxies are embedded \citep[see e.g.][]{Kuutma2017}. A final example of a possible influence of the cosmic web on the nature of 
galaxies concerns a more recent finding that has lead to a vigorous activity in seeking to understand it. The satellite galaxy 
systems around the Galaxy and M31 have been found to be flattened. It might be that their orientation points at a direct influence 
of the surrounding large scale structures \citep[see][]{Ibata2013,Cautun2015,2015MNRAS.452.1052L,Forero-Romero2015,Gonzalez2016}, for example a reflection of local filament or local sheet.


\section{Cosmic fields: Characterizing non-linear growth of Cosmic structure}
% Taken from Comprehensive essay
% /home/nes/MEGA/Google_drive/KU courses/Fall2016/CompreEssay/Compre1/draft_compre_clean.tex


The most fundamental attributes of particles in the N-body simulations are their position and velocity co-ordinates, and their masses. Due to the lack of numerical tools for direct analysis these raw data, fields such as mass density $\rho(\mathbf{x}, t)$, velocity $\mathbf{v}(\mathbf{x}, t)$ or gravitational potential $\phi(\mathbf{x}, t)$ fields are often computed numerically. Mass density fields were calculated using Cloud-in-Cell (CIC) algorithm (cf. \citealt{Hockney1988}), which is numerically equivalent to counting the number of particles on each cell of a regular grid. Alternatively, the density field also generated on irregular grids by applying Delaunay (For example, \citealt{Icke1991} and the Delaunay Tessellation Field Estimator (DTFE) by \citealt{Schaap2000} and \citealt{Weygaert2009a}) or Voronoi tessellations (See \citealt{Schaap2000} and references therein) to the particle coordinates. Another parameter `linking length', using distances between nearest neighbouring particles, was used for percolation analyses and identifying super-clusters of galaxies ( \citealt{Zeldovich1982}, \citealt{Shandarin1983} and \citealt{Shandarin1983b}) for identifying halos \citealt{Davis1985} in the cosmological simulations. Left panel in \autoref{fig:cosmicfields} shows some of the popular fields/parameters that use particle mass and positions. It has to be noted that the density fields or linking-lengths are not dynamical descriptions that invoke the initial field of density fluctuations or the velocity of the particles. 


\begin{figure*}
\begin{minipage}[t]{0.99\linewidth}
 \centering\includegraphics[height=8.5cm]{Chapter1/Plots/fig0.pdf} 
\end{minipage}\hfill
\captionsetup{font={small,stretch=1}}
\caption{Classification of some of the fields/parameters used in cosmological analyses. Some fields utilise position co-ordinates only whereas others use the full phase-space information. In addition, the fields may be defined on a regular grid, or may be defined on an unstructured grid (For instance, flip-flop is a number-valued field defined on each dark matter particle). Fields like mass density and multi-streams can be defined on either grids, depending on the numerical technique. List is obviously not exhaustive- velocity and potential fields are not included, and discussions of correlation functions are excluded as well.}
\label{fig:cosmicfields}
\end{figure*}


An obvious advantage of methods based on particle coordinates, both on structured and unstructured grids, is their applicability to redshift catalogues. The redshift catalogues like SDSS and 2dF provide only two angular coordinates and distances in redshift space. But cosmological N-body dark matter simulations provide the full dynamical information in six-dimensional phase space. This additional information is very valuable providing a greater opportunity for understanding the physics of the web and developing a better theory of the web.


The velocity fields in the simulations of collision-less cold dark matter particles can become multi-valued under the action of gravity. This phenomenon was first discussed by \cite{Zeldovich1970}, where he predicted the formation of non-linear structures (also see \citealt{Shandarin1989} for discussion on formation of multi-stream). The primordial oblate structures were later known as `Zel'dovich pancakes'. These pancakes grow from initial perturbations in a continuous mass distribution, where the velocities are single-valued (also referred to as single-stream) everywhere in the configuration space. Multiple values in the velocity field $\mathbf{v} (\mathbf{x},t)$ or `multi-streams' can also be seen in the dynamically equivalent Lagrangian sub-manifold - $(\mathbf{q}, \mathbf{x})$, where $\mathbf{x}$ and $\mathbf{q}$ are co-moving Eulerian and Lagrangian co-ordinates respectively. \cite{Shandarin2011} and \cite{Abel2012b} studied this $ \mathbf{q} \mapsto \mathbf{x}$ mapping in N-body simulations to quantify the number of streams using phase-space tessellations. \cite{Shandarin2011} define a multi-stream field $n_{str}(\mathbf{x})$ as a field taking discrete values that are equal to the number of streams at every evaluation point in configuration space. Ordered sign-reversal of each elementary volume element in the Lagrangian sub-manifold was measured by \cite{Shandarin2014a}. Their {\it flip-flop} field $n_{ff}(\mathbf{q})$ in Lagrangian space demonstrates a very rich sub-structure of the cosmic web, especially in a halo environment. 

Fields computed from a complete dynamical information (either $(\mathbf{q}, \mathbf{x})$ or $(\mathbf{p}, \mathbf{q})$) could provide valuable contributions to our understanding of the cosmic structure. \cite{Falck2012} have recently delineated archetypal web structures by counting the number of foldings in the sub-manifold for each dark matter particle along different directions. Another study by \cite{Ramachandra2016a} explored some of the global topological and local geometrical properties of the web in the context of multi-streaming. The applications of these analyses is certainly not limited to diagnostic tools; the multi-streaming phenomenon can be used in improving N-body simulations \citep{Hahn2013}, and studying galaxy evolution and star formation as well \citep{Aragon-Calvo2016b}. 

\section{Chapter Organization}